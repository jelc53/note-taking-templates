%------------------------------------------------------------
% lecture notes template
% invariants (macros/preamble) live in macros.tex
%------------------------------------------------------------
\documentclass[11pt,a4paper]{article}

% Code listings (loaded here so it applies to all notes)
\usepackage{listings}
\lstset{
basicstyle=\ttfamily\small,
frame=single,
breaklines=true,
columns=fullflexible,
showstringspaces=false,
numbers=left,
numberstyle=\tiny,
captionpos=b
}

% Shared macros, envrionments and layout

\documentclass{article}

\usepackage{graphicx}

\newcommand{\di}{{d}}
\newcommand{\nexp}{{n}}
\newcommand{\nf}{{p}}
\newcommand{\vcd}{{\textbf{D}}}

\usepackage{nccmath}
\usepackage{mathtools}
\usepackage{graphicx,caption}
\usepackage{enumitem}
\usepackage{epstopdf,subcaption}
\usepackage{psfrag}
\usepackage{amsmath,amssymb,epsf}
\usepackage{verbatim}
\usepackage[hyphens]{url}
\usepackage{color}
\usepackage{bbold}
\usepackage{bbm}
\usepackage{listings}
\usepackage{setspace}
\usepackage{float}
\usepackage{natbib}
\definecolor{Code}{rgb}{0,0,0}
\definecolor{Decorators}{rgb}{0.5,0.5,0.5}
\definecolor{Numbers}{rgb}{0.5,0,0}
\definecolor{MatchingBrackets}{rgb}{0.25,0.5,0.5}
\definecolor{Keywords}{rgb}{0,0,1}
\definecolor{self}{rgb}{0,0,0}
\definecolor{Strings}{rgb}{0,0.63,0}
\definecolor{Comments}{rgb}{0,0.63,1}
\definecolor{Backquotes}{rgb}{0,0,0}
\definecolor{Classname}{rgb}{0,0,0}
\definecolor{FunctionName}{rgb}{0,0,0}
\definecolor{Operators}{rgb}{0,0,0}
\definecolor{Background}{rgb}{0.98,0.98,0.98}
\lstdefinelanguage{Python}{
numbers=left,
numberstyle=\footnotesize,
numbersep=1em,
xleftmargin=1em,
framextopmargin=2em,
framexbottommargin=2em,
showspaces=false,
showtabs=false,
showstringspaces=false,
frame=l,
tabsize=4,
% Basic
basicstyle=\ttfamily\footnotesize\setstretch{1},
backgroundcolor=\color{Background},
% Comments
commentstyle=\color{Comments}\slshape,
% Strings
stringstyle=\color{Strings},
morecomment=[s][\color{Strings}]{"""}{"""},
morecomment=[s][\color{Strings}]{'''}{'''},
% keywords
morekeywords={import,from,class,def,for,while,if,is,in,elif,else,not,and,or
,print,break,continue,return,True,False,None,access,as,,del,except,exec
,finally,global,import,lambda,pass,print,raise,try,assert},
keywordstyle={\color{Keywords}\bfseries},
% additional keywords
morekeywords={[2]@invariant},
keywordstyle={[2]\color{Decorators}\slshape},
emph={self},
emphstyle={\color{self}\slshape},
%
}


\pagestyle{empty} \addtolength{\textwidth}{1.0in}
\addtolength{\textheight}{0.5in}
\addtolength{\oddsidemargin}{-0.5in}
\addtolength{\evensidemargin}{-0.5in}
\newcommand{\ruleskip}{\bigskip\hrule\bigskip}
\newcommand{\nodify}[1]{{\sc #1}}
\newcommand{\points}[1]{{\textbf{[#1 points]}}}
\newcommand{\subquestionpoints}[1]{{[#1 points]}}
\newenvironment{answer}{{\bf Answer:} \sf \begingroup\color{red}}{\endgroup}%

\newcommand{\bitem}{\begin{list}{$\bullet$}%
{\setlength{\itemsep}{0pt}\setlength{\topsep}{0pt}%
\setlength{\rightmargin}{0pt}}}
\newcommand{\eitem}{\end{list}}

\setlength{\parindent}{0pt} \setlength{\parskip}{0.5ex}
\setlength{\unitlength}{1cm}

\renewcommand{\Re}{{\mathbb R}}
\newcommand{\R}{\mathbb{R}}
\newcommand{\what}[1]{\widehat{#1}}

\renewcommand{\comment}[1]{}
\newcommand{\mc}[1]{\mathcal{#1}}
\newcommand{\half}{\frac{1}{2}}

\def\KL{D_{KL}}
\def\xsi{x^{(i)}}
\def\ysi{y^{(i)}}
\def\zsi{z^{(i)}}
\def\E{\mathbb{E}}
\def\calN{\mathcal{N}}
\def\calD{\mathcal{D}}

\usepackage{tikz}
\usepackage{bbding}
\usepackage{pifont}
\usepackage{wasysym}
\usepackage{amssymb}
\usepackage{booktabs}
\usepackage{verbatim}



\begin{document}

% ====== PER-LECTURE HEADER (edit these each time) ======
\lecture{1}{Introduction to Limit Theory}{2025-09-15}{Electrical Engineering 101}{Terrance Tao}

% ====== SHORT, ONE-PAGE SAMPLE ======
\section*{Overview}
\begin{itemize}
\item Notation: \(\lim_{x\to a} f(x)=L\) indicates that values of \(f(x)\) approach \(L\) as \(x\) approaches \(a\).
\item Intuition: closeness of \(x\) to \(a\) forces closeness of \(f(x)\) to \(L\).
\item Use algebraic simplification first when limits yield indeterminate forms such as \(0/0\).
\end{itemize}


\section*{Definition}
For a function \(f: \R\to\R\), we say \(\lim_{x\to a} f(x)=L\) if for every \(\varepsilon>0\) there exists \(\delta>0\) such that whenever \(0<|x-a|<\delta\), it follows that \(|f(x)-L|<\varepsilon\).


\section*{Example}
Evaluate \(\displaystyle \lim_{x\to 2} \frac{x^2-4}{x-2}.\)


\noindent\textit{Solution.} Since \(x^2-4=(x-2)(x+2)\), for \(x\neq 2\) we have \(\frac{x^2-4}{x-2}=x+2\). Hence the limit equals \(\lim_{x\to 2}(x+2)=4\). \hfill(\emph{removable discontinuity})


\section*{Numerical verification (script)}
\begin{lstlisting}[language=Python, caption={Two-sided evaluation near \(x=2\)}]
import math


def f(x):
return (x**2 - 4) / (x - 2)


for h in [1e-1, 1e-2, 1e-3, 1e-4]:
print(f"x=2+h -> {f(2+h):.6f} x=2-h -> {f(2-h):.6f}")
# Expected output: both sequences approach 4
\end{lstlisting}


\section*{Key takeaways}
\begin{itemize}
\item The \(\varepsilon\)–\(\delta\) definition formalizes the intuitive notion of approach.
\item Simplification often eliminates indeterminate forms.
\item A limit may exist even when \(f\) is undefined at the point.
\end{itemize}


\end{document}
